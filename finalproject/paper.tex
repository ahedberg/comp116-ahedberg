\documentclass[pdftex,letterpaper,titlepage,11pt]{article}
\usepackage[margin=1in]{geometry}
\usepackage{hyperref}
\usepackage{array}

\hypersetup{
  colorlinks=false,
  pdfborder={0 0 0}
}

\begin{document}
  \title{The Privacy of Private Browsing}
  \author{Ashley Hedberg\thanks{\emph{Mentor:} Ming Chow, Tufts University} \\
  \href{mailto:ashley.hedberg@tufts.edu}{ashley.hedberg@tufts.edu}}
  \date{}
  \maketitle
  
  \begin{abstract}
  Eventually the abstract will go here.
  \end{abstract}

  \section{Introduction}
    \subsection{Browser Overview}
    The following table summarizes the web browsers that will be considered in
    this paper. The features that a particular web browser's private mode is
    believed to have comes from the developers of the browser. Note that all
    browsers run on both Windows and Linux, with the exception of Internet
    Explorer, which runs only on Windows.\cite{verdi13}\cite{google13}
    \cite{ie13}

    \begin{center}
      \begin{tabular}{|m{'5cm'}|c|c|c|}
        \hline
        Property & Mozilla Firefox & Google Chrome & Internet Explorer \\
        \hline
        Browser history & X & X & X \\
        \hline
        Form data & X &  & X \\
        \hline
        Search bar data & X &  & X \\
        \hline
        Passwords & X &  & X \\
        \hline
        Downloads & X & X &  \\
        \hline
        Cookies & X & X & X \\
        \hline
        Cached files & X &  & X \\
        \hline
        Simulataneous sessions with different privacy settings & X & X & X \\
        \hline
        Hides activity from ISP &  &  &  \\
        \hline
        Hides activity from websites &  &  &  \\
        \hline
      \end{tabular}
    \end{center}

  This paper also discusses the Tor browser. Unlike Firefox, Chrome, and
  Internet Explorer, the Tor browser connects users to the internet using the
  Tor network. This does much more to preserve browsing anonymity than the
  browsers listed above. TODO finish this\ldots lazy. 

  Most modern web browsers offer some sort of private browsing mode. This is
  supposed to allow users to view web content without other users of the same
  computer being able to determine what websites were visited. What is kept
  private and what is not differs from browser to browser. An overview of the
  most common browsers for Windows and Linux is provided below.

    \subsection{Mozilla Firefox}
    Mozilla Firefox runs on both Windows and Linux operating systems. Its
    private browsing mode is known as simply Private Browsing. Mozilla explains
    that it will not save information about websites and web pages accessed in
    a private browsing session, but it also clearly states that it does not
    preserve internet anonymity and does not prevent internet service
    providers, network administrators, or websites from recording such
    information. It is possible to have both private and non-private windows
    open at the same time.\cite{verdi13}

    \subsection{Google Chrome}
    Google Chrome also runs on both Windows and Linux. Its private browsing
    mode is called incognito mode. Browsing history and downloads are not
    recorded when browsing in incognito mode, and any cookies created during
    an incognito browsing session are deleted once the last incognito window is
    closed. As with Firefox, private and non-private windows can be open at the
    same time.\cite{google13}

    \subsection{Microsoft Internet Explorer}
    Microsoft Internet Explorer only runs on Windows. Its private browsing mode
    is called InPrivate Browsing. Microsoft claims that this browsing mode
    helps to ``prevent'' pages visited, cached data, cookies, and login
    information from being stored after the browsing session terminates, but it
    only guarantees that an InPrivate Browsing session won't retain information
    about websites accessed or web searches. Once again, private and non-
    private windows can be open simultaneously.\cite{ie13}

    \subsection{Tor Browser Bundle}
    The Tor Browser Bundle runs on both Windows and Linux. 

  \section{To the Community}

  \section{Applications}

  \section{Conclusion}

  \begin{thebibliography}{9}
    \bibitem{verdi13}
      Verdi, Michael et al. ``Private Browsing.'' \emph{Mozilla Support}. 
      Mozilla Foundation, 29 Mar. 2013. Web. 10 Dec. 2013. 
      $<$\url{http://support.mozilla.org/en-US/kb/private-browsing-browse-web-
      without-saving-info}$>$.
    \bibitem{google13}
      ``Chrome Browser.'' \emph{Chrome}. Google, 18 Nov. 2013. Web. 10 Dec. 
      2013. $<$\url{http://www.google.com/intl/en/chrome/browser/
      features.html#privacy}$>$.
    \bibitem{ie13}
      ``InPrivate Browsing.'' \emph{Microsoft Windows}. Microsoft, 10 Dec. 
      2013. Web. 10 Dec. 2013. $<$\url{http://windows.microsoft.com/en-us/
      internet-explorer/products/ie-9/features/in-private}$>$.
    \bibitem{tor13}
      ``Tor Project: Overview.'' \emph{Tor}. Tor, 7 Dec. 2013. Web. 10 Dec. 
      2013. $<$\url{http://www.torproject.org/about/overview.html.en}$>$. 

  \end{thebibliography}


\end{document}
