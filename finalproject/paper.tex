\documentclass[pdftex,letterpaper,titlepage,12pt]{article}
\usepackage[margin=1in]{geometry}
\usepackage{hyperref}
\usepackage[doublespacing]{setspace}

\hypersetup{
  colorlinks=false,
  pdfborder={0 0 0}
}
\setlength\parindent{0.5in}

\begin{document}
  \singlespacing
  \title{The Privacy of Private Browsing}
  \author{Ashley Hedberg\thanks{\emph{Mentor:} Ming Chow, Tufts University} \\
  \href{mailto:ashley.hedberg@tufts.edu}{ashley.hedberg@tufts.edu}}
  \date{}
  \maketitle
  
  \begin{abstract}
  Most modern web browsers have a ``private browsing'' mode that supposedly 
  allows a user to surf the internet without leaving any traces of his or her
  activity on his or her machine. However, the notion of ``private browsing'' 
  offers users a false sense of security, as browsing information is often left
  behind when a private browsing session terminates. Several researchers have 
  already demonstrated methods of detecting this information. These include the
  analysis of virtual and browser memory and the pagefile on Windows machines.
  The existence of private browsing artifacts on a user’s machine raises many 
  questions: how much privacy do web browser developers actually aim to provide
  with ``private browsing'' modes? Are these goals accurately conveyed to the 
  end user? And, if the user’s browsing session can be reconstructed from these
  artifacts, how might this be exploited by digital forensics professionals? 
  This paper seeks to answer these questions and to prove that such artifacts 
  can be used to determine, at least in part, what a user was doing during his 
  or her ``private browsing'' session---thereby rendering it not very private 
  at all.
  \end{abstract}

  \doublespacing
  \section{Introduction}
    \subsection{Browser Overview}
    Currently, the most popular browsers for the Windows family of operating 
    systems are Mozilla Firefox, Google Chrome, and Microsoft Internet 
    Explorer. All three of these browsers have built-in private browsing modes.
    The features and protections provided by each of these private browsing
    modes are summarized below.\cite{verdi13}\cite{google13}\cite{ie13}

    \singlespacing
    \begin{center}
      \begin{tabular}{|p{5cm}|c|c|c|}
        \hline
         & Firefox & Chrome & Internet Explorer \\
         & Private Browsing & Incognito Mode & InPrivate Browsng \\
        \hline
        Browser history & X & X & X \\
        \hline
        Form data & X &  & X \\
        \hline
        Search bar data & X &  & X \\
        \hline
        Passwords & X &  & X \\
        \hline
        Downloads & X & X &  \\
        \hline
        Cookies & X & X & X \\
        \hline
        Cached files & X &  & X \\
        \hline
        Simulataneous sessions with different privacy settings & X & X & X \\
        \hline
      \end{tabular}
    \end{center}
    \doublespacing

    \subsection{What Private Browsing Is Not}
    The private browsing modes of Firefox, Chrome, and Internet Explorer are
    designed with the intent of preventing users of the same computer from
    determining what someone was doing on that computer. They do not attempt
    to keep a user's internet activities hidden from that user's internet
    service provider (ISP), and they do not prevent websites from identifying
    that user.\cite{verdi13}\cite{google13}\cite{ie13} Users seeking this kind 
    of anonymity should use the Tor browser, which employs onion routing to 
    connect a user to a destination server. In this type of routing, a random 
    path from the user to the destination is constructed, and content is 
    encrypted at each point in the path. Furthermore, each point (known as a 
    relay or node) only knows from which relay any given piece of network 
    traffic immediately came and the next relay to which it is going. It does 
    not know the original source of the network traffic or its final 
    destination. This is what allows users to keep themselves and their 
    activities anonymous using Tor in a way that the traditional web browsers 
    cannot.\cite{tor13}

    The specifics of the Tor network and how it can be used (and abused) will
    not be considered here. It is mentioned because the Tor browser, like the
    more common web browsers, can leave artifacts of web browsing sessions on
    an individual's computer.
    
  \section{To the Community}
  There are many misconceptions about private browsing. Some believe that it
  prevents an internet service provider, network administrator, or attackers
  engaged in packet sniffing from linking a user's internet activities to his
  or her identity. It does not. Any and all network packets leaving the user's
  machine contain information such as the user's IP address, which can be used
  to determine his or her location and/or identity. Others think that private
  browsing sessions leave absolutely no trace of their activities on their
  local machines. As the next section of this paper will describe, this is also
  false. Still others think that private browsing will prevent the National
  Security Agency from tracking them on social media. It certainly does not.
  This paper will shed some light on what private browsing sessions can and
  cannot do, allowing internet users to think twice before exploring the web.

  \section{Applications: Forensics}
  One of the biggest problems with private browsing sessions is that artifacts
  remain on a machine after the user exits his or her internet browser. These
  artifacts can include logon information for websites, browsing history,
  and digital media such as images and videos---all items that you would expect
  a private browsing session to keep private.\cite{ohana13}

  This section will first summarize locations on a Windows machine where
  private browsing artifacts might be found. An overview of what specific
  artifacts were found for each browser using different forensic techniques
  will follow. The use of SQLite to investigate specific browser artifacts will
  then be discussed.

    \subsection{Where To Look}
    

    Focused analysis of the hard drive itself can reveal the most artifacts.
    One private browsing experiment in which logon information, browser
    history, and cached images were recovered obtained most of the artifacts
    from portions of the hard disk known as \emph{free space} and \emph{slack
    space}. Free space refers to a group of hard drive sectors (called a
    \emph{cluster} that is not allocated to any particular file. Slack space, 
    on the other hand, refers to the space between the end of a file and the 
    end of the cluster in which it is stored. Hard drive clusters are of a 
    fixed size determined by the operating system. Most of the time, when a 
    user saves a file to his or her hard drive, the file is not exactly the
    same size as the cluster in which the operating system puts it.
    Furthermore, when a user ``deletes'' a file from his or her hard drive, the
    file is not really deleted---it is simply marked as available to be
    overwritten if necessary. If a file is eventually stored on the hard drive
    in a cluster that used to be home to a larger (now ``deleted'') file,
    leftover data from the old file will still exist at the end of that
    cluster.

    \subsection{Firefox}


    \subsection{Google Chrome}


    \subsection{Internet Explorer}
    Internet Explorer leaves the most artifacts behind when its InPrivate
    Browsing mode terminates.\cite{ohana13} 

    \subsection{Tor}


    \subsection{Forensic Software}

  \section{Conclusion}


  \singlespacing
  \begin{thebibliography}{9}
    \bibitem{verdi13}
      Verdi, Michael et al. ``Private Browsing.'' \emph{Mozilla Support}. 
      Mozilla Foundation, 29 Mar. 2013. Web. 10 Dec. 2013. 
      $<$\url{http://support.mozilla.org/en-US/kb/private-browsing-browse-web-
      without-saving-info}$>$.
    \bibitem{google13}
      ``Chrome Browser.'' \emph{Chrome}. Google, 18 Nov. 2013. Web. 10 Dec. 
      2013. $<$\url{http://www.google.com/intl/en/chrome/browser/
      features.html#privacy}$>$.
    \bibitem{ie13}
      ``InPrivate Browsing.'' \emph{Microsoft Windows}. Microsoft, 10 Dec. 
      2013. Web. 10 Dec. 2013. $<$\url{http://windows.microsoft.com/en-us/
      internet-explorer/products/ie-9/features/in-private}$>$.
    \bibitem{tor13}
      ``Tor Project: Overview.'' \emph{Tor}. Tor, 7 Dec. 2013. Web. 10 Dec. 
      2013. $<$\url{http://www.torproject.org/about/overview.html.en}$>$. 
    \bibitem{ohana13}
      Ohana, Donny, and Narasimha Shashidhar. ``Do Private and Portable Web 
      Browsers Leave Incriminating Evidence? A Forensic Analysis of Residual 
      Artifacts from Private and Portable Web Browsing Sessions.'' IEEE CS 
      Security and Privacy Workshops (SPW), The Westin St. Francis, 
      San Francisco, CA. 23-24 May 2013. Web. 9 Dec. 2013. $<$\url{http://www.
      ieee-security.org/TC/SPW2013/papers/data/5017a135.pdf}$>$

  \end{thebibliography}


\end{document}
