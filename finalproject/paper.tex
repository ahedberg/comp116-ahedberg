\documentclass[pdftex,letterpaper,titlepage,12pt]{article}
\usepackage[margin=1in]{geometry}
\usepackage{hyperref}
\usepackage[doublespacing]{setspace}

\hypersetup{
  colorlinks=false,
  pdfborder={0 0 0}
}
\setlength\parindent{0.5in}

\begin{document}
  \singlespacing
  \title{The Privacy of Private Browsing}
  \author{Ashley Hedberg\thanks{\emph{Mentor:} Ming Chow, Tufts University} \\
  \href{mailto:ashley.hedberg@tufts.edu}{ashley.hedberg@tufts.edu}}
  \date{}
  \maketitle
  
  \begin{abstract}
  Most modern web browsers have a ``private browsing'' mode that supposedly 
  allows a user to surf the internet without leaving any traces of his or her
  activity on his or her machine. However, the notion of ``private browsing'' 
  offers users a false sense of security, as browsing information is often left
  behind when a private browsing session terminates. Several researchers have 
  already demonstrated methods of detecting this information. These include the
  analysis of virtual and browser memory and the pagefile on Windows machines.
  The existence of private browsing artifacts on a user’s machine raises many 
  questions: how much privacy do web browser developers actually aim to provide
  with ``private browsing'' modes? Are these goals accurately conveyed to the 
  end user? And, if the user’s browsing session can be reconstructed from these
  artifacts, how might this be exploited by digital forensics professionals? 
  This paper seeks to answer these questions and to prove that such artifacts 
  can be used to determine, at least in part, what a user was doing during his 
  or her ``private browsing'' session---thereby rendering it not very private 
  at all.
  \end{abstract}

  \doublespacing
  \section{Introduction}
    \subsection{Browser Overview}
    Currently, the most popular browsers for the Windows family of operating 
    systems are Mozilla Firefox, Google Chrome, and Microsoft Internet 
    Explorer. All three of these browsers have built-in private browsing modes.
    The features and protections provided by each of these private browsing
    modes are summarized below.\cite{verdi13}\cite{google13}\cite{ie13}

    \singlespacing
    \begin{center}
      \begin{tabular}{|p{5cm}|c|c|c|}
        \hline
         & Firefox & Chrome & Internet Explorer \\
         & Private Browsing & Incognito Mode & InPrivate Browsng \\
        \hline
        Browser history & X & X & X \\
        \hline
        Form data & X &  & X \\
        \hline
        Search bar data & X &  & X \\
        \hline
        Passwords & X &  & X \\
        \hline
        Downloads & X & X &  \\
        \hline
        Cookies & X & X & X \\
        \hline
        Cached files & X &  & X \\
        \hline
        Simulataneous sessions with different privacy settings & X & X & X \\
        \hline
      \end{tabular}
    \end{center}
    \doublespacing
	
	\subsection{Definitions}
	Many shortcomings of private browsing modes relate to how a computer's 
	operating system manages memory and disk space. Those who are unfamiliar
	with these tasks may benefit from the definitions below.
	
	\begin{description}
	  \item[Cluster] A group of disk sectors that the operating system
	  treats as a unit; smallest unit of disk storage that the operating
	  system can allocate
	  \item[Disk] The hard drive of a computer; where a user's files and
	  directories (among other things) are stored
	  \item[Free space] A cluster that is not allocated to any particular 
	  file
	  \item[Page] The smallest unit of memory that the operating system can
	  allocate; can be temporarily stored on disk to allow the computer to run
	  more programs at the same time
	  \item[Slack space] The space between the end of a file and the end of 
	  the cluster in which it is stored; important in computer forensics for
	  finding fragments of old files\footnote{Hard drive clusters are of a 
	  fixed size determined by the operating system. Most of the time, when a 
	  user saves a file to his or her hard drive, the file is not exactly the 
	  same size as the cluster in which the operating system puts it. 
	  Furthermore, when a user ``deletes'' a file from his or her hard drive, 
	  the file is not really deleted---it is simply marked as available to be
      overwritten if necessary. If a file is eventually stored on the hard 
	  drive in a cluster that used to be home to a larger (now ``deleted'') 
	  file, leftover data from the old file will still exist at the end of that
      cluster.}
	\end{description}
    
  \section{To the Community}
  There are many misconceptions about private browsing. Some believe that it
  prevents an internet service provider, network administrator, or attackers
  engaged in packet sniffing from linking a user's internet activities to his
  or her identity. It does not. Any and all network packets leaving the user's
  machine contain information such as the user's IP address, which can be used
  to determine his or her location and/or identity. Others think that private
  browsing sessions leave absolutely no trace of their activities on their
  local machines. As the next section of this paper will describe, this is also
  false. Still others think that private browsing will prevent the National
  Security Agency from tracking them on social media. It certainly does not.
  This paper will shed some light on what private browsing sessions can and
  cannot do, allowing internet users to think twice before exploring the web.

  \section{Applications: Forensics}
  One of the biggest problems with private browsing sessions is that artifacts
  remain on a machine after the user exits his or her internet browser. These
  artifacts can include logon information for websites, browsing history,
  and digital media such as images and videos---all items that you would expect
  a private browsing session to keep private.\cite{ohana13}

  This section will first summarize locations on a Windows machine where
  private browsing artifacts might be found. An overview of what specific
  artifacts were found for each browser using different forensic techniques
  will follow. The use of SQLite to investigate specific browser artifacts will
  then be discussed.

    \subsection{Where To Look}
    Focused analysis of the hard drive itself can reveal the most artifacts.
    One private browsing experiment in which logon information, browser
    history, and cached images were recovered obtained most of the artifacts
    from the hard drive's free space and slack space.\cite{ohana13}

    Artifacts can also be recovered from the computer's memory itself. Cached
    web documents can often be found here.\cite{ohana13} Browsing history,
    logon information, and cookies have been detected after private browsing
    sessions by analyzing a computer's memory. The SQLite databases used by 
    Chrome and Firefox during browsing sessions are generally only stored in 
    memory when private browsing is in use, and their residual data could be 
    detected, as well.\cite{satvat13}

    The Windows pagefile can also reveal private browsing artifacts. This file
    is where the least recently used pages of memory are stored when too many
    applications are competing for the computer's physical memory. Browsing
    history and keywords used in internet search engines have been discovered
    here.\cite{said11}

    \subsection{Firefox}
	In 2011, researchers discovered that Firefox 3.6.11 stored browser history
	and search engine keywords in the computer's physical memory and that this
	information could be accessed after the browsing session was terminated.
	An analysis of the contents of pagefile.sys revealed similar information.
	\cite{said11} Additionally, certificate chains received during private 
	browsing sessions in Firefox 3.6 were not removed from their on-disk 
	database once the browsing session was terminated. 

    \subsection{Google Chrome}


    \subsection{Internet Explorer}
    Internet Explorer leaves the most artifacts behind when its InPrivate
    Browsing mode terminates.\cite{ohana13} 

    \subsection{Forensic Software}


  \section{Conclusion}
  With all the ways in which private browsing modes can leak information, it
  may seem as though they are completely useless. This is not entirely true.
  While private browsing sessions won't help anyone evade law enforcement, a
  forensic investigation, or the United States government, they are still good
  enough to fool most computer users. Nosy roommates, family members, or
  patrons at the public library are unlikely to go to the lengths discussed in
  this paper to determine what someone else was doing on the internet.

  That said, private browsing modes don't do much to keep a user's identity
  private to the outside world. Users seeking this kind of anonymity should 
  consider using the Tor browser, which employs onion routing to connect them 
  to their destination server. In this type of routing, a random path from the 
  user to the destination is constructed, and content is encrypted at each 
  point in the path. Furthermore, each point (known as a relay or node) only 
  knows from which relay any given piece of network traffic immediately came 
  and the next relay to which it is going. It does not know the original source
  of the network traffic or its final destination. This is what allows users to
  keep themselves and their activities anonymous using Tor in a way that the 
  traditional web browsers cannot.\cite{tor13} Of course, this type of browsing
  can be (ab)used for a whole host of illicit activities, but those are beyond
  the scope of this paper.

  Until Mozilla, Google, and Microsoft reduce the artifacts leaker by their
  respective browsers' private modes, users who are concerned about ``local 
  attackers''---those of the nosy roommate or family member variety who have 
  physical access to their machine---can do their part to keep their private
  information secure. Random access memory, where private browsing artifacts
  have been found, is cleared when a computer is powered down. Shutting down
  the computer after a private browsing session can reduce or eliminate these
  artifacts. Hard drive artifacts, however, are nearly impossible to eradicate.
  The moral of the story is that private browsing is nothing more than a nice
  illusion useful for fooling the computer-illiterate. At the end of the day,
  private browsing really isn't that private at all.

  \singlespacing
  \begin{thebibliography}{9}
    \bibitem{verdi13}
      Verdi, Michael et al. ``Private Browsing.'' \emph{Mozilla Support}. 
      Mozilla Foundation, 29 Mar. 2013. Web. 10 Dec. 2013. 
      $<$\url{http://support.mozilla.org/en-US/kb/private-browsing-browse-web-
      without-saving-info}$>$.
    \bibitem{google13}
      ``Chrome Browser.'' \emph{Chrome}. Google, 18 Nov. 2013. Web. 10 Dec. 
      2013. $<$\url{http://www.google.com/intl/en/chrome/browser/
      features.html#privacy}$>$.
    \bibitem{ie13}
      ``InPrivate Browsing.'' \emph{Microsoft Windows}. Microsoft, 10 Dec. 
      2013. Web. 10 Dec. 2013. $<$\url{http://windows.microsoft.com/en-us/
      internet-explorer/products/ie-9/features/in-private}$>$.
    \bibitem{tor13}
      ``Tor Project: Overview.'' \emph{Tor}. Tor, 7 Dec. 2013. Web. 10 Dec. 
      2013. $<$\url{http://www.torproject.org/about/overview.html.en}$>$. 
    \bibitem{ohana13}
      Ohana, Donny, and Narasimha Shashidhar. ``Do Private and Portable Web 
      Browsers Leave Incriminating Evidence? A Forensic Analysis of Residual 
      Artifacts from Private and Portable Web Browsing Sessions.'' IEEE CS 
      Security and Privacy Workshops (SPW), The Westin St. Francis, 
      San Francisco, CA. 23-24 May 2013. Web. 9 Dec. 2013. $<$\url{http://www.
      ieee-security.org/TC/SPW2013/papers/data/5017a135.pdf}$>$
    \bibitem{satvat13}
      Satvat, Kiavash, Matthew Forshaw, Feng Hao, and Ehsan Toreini. ``On The 
      Privacy of Private Browsing - A Forensic Approach.'' Proceedings of the 
      8th International Workshop on Data Privacy Management (DPM '13), Royal 
      Holloway, University of London, Egham, UK. 12-13 Sept. 2013. Web. 9 Dec.
      2013. $<$\url{http://homepages.cs.ncl.ac.uk/feng.hao/files/DPM13.pdf}$>$ 
    \bibitem{said11}
      Said, Huwida, Noora Al Mutawa, Ibtesam Al Awadhi, and Mario Guimaraes.
      ``Forensic Analysis of Private Browsing Artifacts.'' 7th International
      Conference on Innovations in Information Technology, Abu Dhabi, United
      Arab Emirates. 25-27 Apr. 2011. Web. 10 Dec. 2013. $<$\url{http://
      ieeexplore.ieee.org/xpl/articleDetails.jsp?tp=&arnumber=5893816}$>$ 

  \end{thebibliography}


\end{document}
