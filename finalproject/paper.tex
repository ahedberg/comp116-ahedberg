\documentclass[pdftex,letterpaper,12pt]{article}
\usepackage[margin=1in]{geometry}
\usepackage{hyperref}
\usepackage[doublespacing]{setspace}

\hypersetup{
  colorlinks=false,
  pdfborder={0 0 0}
}
\setlength\parindent{0.5in}

\begin{document}
  \singlespacing
  \title{The Privacy of Private Browsing}
  \author{Ashley Hedberg\thanks{\emph{Mentor:} Ming Chow, Tufts University} \\
  \href{mailto:ashley.hedberg@tufts.edu}{ashley.hedberg@tufts.edu}}
  \date{}
  \maketitle
  
  \begin{abstract}
  Most modern web browsers have a ``private browsing'' mode that supposedly 
  allows a user to surf the internet without leaving any traces of his or her
  activity on his or her machine. However, the notion of ``private browsing'' 
  offers users a false sense of security, as browsing information is often left
  behind when a private browsing session terminates. Several researchers have 
  already demonstrated methods of detecting this information. These include the
  analysis of virtual and browser memory and the pagefile on Windows machines.
  The existence of private browsing artifacts on a user’s machine raises many 
  questions: how much privacy do web browser developers actually aim to provide
  with ``private browsing'' modes? Are these goals accurately conveyed to the 
  end user? And, if the user’s browsing session can be reconstructed from these
  artifacts, how might this be exploited by digital forensics professionals? 
  This paper seeks to answer these questions and to prove that such artifacts 
  can be used to determine, at least in part, what a user was doing during his 
  or her ``private browsing'' session---thereby rendering it not very private 
  at all.
  \end{abstract}

  \doublespacing
  \section{Introduction}
    \subsection{Browser Overview}
    Currently, the most popular browsers for the Windows and Linux families of
    operating systems are Mozilla Firefox, Google Chrome, and Microsoft 
    Internet Explorer. (Internet Explorer runs only on the Windows family of
    operating systems, whereas Firefox and Chrome run on both Windows and 
    Linux.) All three of these browsers have built-in private browsing modes.
    The features and protections provided by each of these private browsing
    modes are summarized below.\cite{verdi13}\cite{google13}\cite{ie13}

    \singlespacing
    \begin{center}
      \begin{tabular}{|p{5cm}|c|c|c|}
        \hline
         & Firefox & Chrome & Internet Explorer \\
         & Private Browsing & Incognito Mode & InPrivate Browsng \\
        \hline
        Browser history & X & X & X \\
        \hline
        Form data & X &  & X \\
        \hline
        Search bar data & X &  & X \\
        \hline
        Passwords & X &  & X \\
        \hline
        Downloads & X & X &  \\
        \hline
        Cookies & X & X & X \\
        \hline
        Cached files & X &  & X \\
        \hline
        Simulataneous sessions with different privacy settings & X & X & X \\
        \hline
      \end{tabular}
    \end{center}
    \doublespacing

    The private browsing modes of Firefox, Chrome, and Internet Explorer are
    designed with the intent of preventing users of the same computer from
    determining what someone was doing on that computer. They do not attempt
    to keep a user's internet activities hidden from that user's internet
    service provider (ISP), and they do not prevent websites from identifying
    that user.\cite{verdi13}\cite{google13}\cite{ie13}

    Users seeking this kind of anonymity should use the Tor browser, which
    employs onion routing to connect a user to a destination server. In this
    type of routing, a random path from the user to the destination is
    constructed, and content is encrypted at each point in the path.
    Furthermore, each point (known as a relay or node) only knows from which
    relay any given piece of network traffic immediately came and the next
    relay to which it is going. It does not know the original source of the
    network traffic or its final destination. This is what allows users to keep
    themselves and their activities anonymous using Tor in a way that the 
    traditional web browsers cannot.\cite{tor13}

    The specifics of the Tor network and how it can be used (and abused) will
    not be considered here. It is mentioned because the Tor browser, like the
    more common web browsers, can leave artifacts of web browsing sessions on
    an individual's computer.
    
    \subsection{Known Issues with Private Browsing}
    Prior research in this area indicates that web browsers, even when used in
    their respective private browsing modes, can leave artifacts on a user's
    machine that can reveal their internet activities. %finish

  \section{To the Community}

  \section{Applications}

  \section{Conclusion}

  \singlespacing
  \begin{thebibliography}{9}
    \bibitem{verdi13}
      Verdi, Michael et al. ``Private Browsing.'' \emph{Mozilla Support}. 
      Mozilla Foundation, 29 Mar. 2013. Web. 10 Dec. 2013. 
      $<$\url{http://support.mozilla.org/en-US/kb/private-browsing-browse-web-
      without-saving-info}$>$.
    \bibitem{google13}
      ``Chrome Browser.'' \emph{Chrome}. Google, 18 Nov. 2013. Web. 10 Dec. 
      2013. $<$\url{http://www.google.com/intl/en/chrome/browser/
      features.html#privacy}$>$.
    \bibitem{ie13}
      ``InPrivate Browsing.'' \emph{Microsoft Windows}. Microsoft, 10 Dec. 
      2013. Web. 10 Dec. 2013. $<$\url{http://windows.microsoft.com/en-us/
      internet-explorer/products/ie-9/features/in-private}$>$.
    \bibitem{tor13}
      ``Tor Project: Overview.'' \emph{Tor}. Tor, 7 Dec. 2013. Web. 10 Dec. 
      2013. $<$\url{http://www.torproject.org/about/overview.html.en}$>$. 

  \end{thebibliography}


\end{document}
